\usepackage[utf8]{inputenc}
\usepackage[T1]{fontenc}
\usepackage[a4paper,left=2cm,right=2cm,top=2cm,bottom=2cm,headheight = 12pt]{geometry}
\usepackage{libertine}
\usepackage[pdftex]{graphicx}
\usepackage{totpages}

\usepackage{helvet}
\renewcommand{\familydefault}{\sfdefault}
\usepackage[english, french]{babel}

\usepackage{setspace}
\onehalfspacing

% Change l'espacement entre deux paragraphes
\setlength{\parskip}{24pt}

% Change l'en-tête et le pied de page pour tout le document
\usepackage{fancyhdr}
\pagestyle{fancy}

\renewcommand{\chaptermark}[1]{\markboth{\thechapter.\ #1}{}}
\renewcommand{\sectionmark}[1]{\markright{\thesection.\ #1}}

\renewcommand{\headrulewidth}{0.5pt} 
\fancyhead[L]{\bfseries\leftmark}
\fancyhead[C]{}
\fancyhead[R]{\rightmark}

\renewcommand{\footrulewidth}{0pt}
\fancyfoot[L]{\textbf{Tugdual Le Pen}}
\fancyfoot[C]{}
\fancyfoot[R]{\thepage\ / \ref{TotPages}}

\fancypagestyle{plain}{ %
    \fancyhf{} % remove everything

    \renewcommand{\headrulewidth}{0pt} 
    \renewcommand{\footrulewidth}{0pt}
    \fancyfoot[L]{\textbf{Tugdual Le Pen}}
    \fancyfoot[C]{}
    \fancyfoot[R]{\thepage\ / \ref{TotPages}}
}

% Change le style des parties et sous partiess
\usepackage[explicit]{titlesec}
% change l'espacement au niveau du titre des chapitres
\titlespacing*{\chapter}
  {0pt}%  indent
  {0pt}% space before
  {12pt}% space after

% Définie les couleurs utilisées dans le rapport
\usepackage{color}

\definecolor{darkblue}{rgb}{0.11, 0.30, 0.66}
\definecolor{lightblue}{rgb}{0.14, 0.38, 0.82}
\definecolor{gray}{rgb}{0.4,0.4,0.4}

%% Style Chapitre
% -------------------------------------------------------
% Avec numéro
\titleformat{\chapter}[hang] 
    {\fontsize{24pt}{0pt}\selectfont \bfseries}
    {\textcolor{darkblue} 
    {\thechapter. #1}}
    {0pt}
    {\huge}

% Sans numéro
\titleformat{name=\chapter,numberless}[hang] 
    {\fontsize{24pt}{0pt}\selectfont \bfseries}
    {\textcolor{darkblue} 
    {#1}}
    {0pt}
    {\huge}
% -------------------------------------------------------

%% Style Section
% -------------------------------------------------------
% Avec numéroté
\titleformat{\section}[hang]
    {\fontsize{18pt}{0pt}\selectfont \bfseries}
    {\textcolor{lightblue} 
    {\thesection.\ #1}}
    {10pt}
    {\Large}

% Sans numéro
\titleformat{name=\section,numberless}[display] 
    {\fontsize{18pt}{0pt}\selectfont \bfseries}
    {\textcolor{lightblue} 
    {\thechapter.\thesection #1}}
    {10pt}
    {\Large}
% -------------------------------------------------------

% Numérotation des chapitre en chiffre Romain
\renewcommand{\thechapter}{\Roman{chapter}}

