\chapter{Introduction}

% rapide résumé contexte, explication du sujet et pk avoir choisi le stage

\section{Contexte}

\par
La peinture et le mouvement du regard de l'Homme ont toujours eu un lien étroit. En effet chaque spectateur regardera un tableau d'une manière différente de son voisin parce que chaque individu à sa propre culture, son propre point de vue, ... Pourtant la structure d'une peinture amènera le spectateur a suivre un sens de lecture. Celui-ci sera généralement commun à tous les spectateur. Par exemple un individu qui découvre le tableau de La Joconde de Leonard de Vinci pour la première fois regardera très souvent en tout premier le visage de Mona Lisa et particulièrement les yeux qui ont un effet particulier. Rare sont les personnes qui commenceront par identifier les éléments du décor en arrière-plan de la peinture.

\par
Ce sont l'ensemble de ces éléments qui attirent l'oeil humain qui consitue la saillance. C'est un élément important pour de nombreux domaines. On pense notamment au domaine du marketing et de la publicité qui doivent créer des affiches ou des spots publicitaires avec pour objectif d'attirer le plus possible le regard.

\par
la saillance dans la peinture permet d'analyser et de comprendre le regard humain ainsi que toutes les particularités qui en découlent. L'équipe Percept, équipe de recherche du laboratoire de l'IRISA, se penche sur le sujet et notamment à l'automatisation pour déterminer la saillance dans les peintures à l'aide de modèles basé sur le machine learning.

\par
C'est là que le sujet de mon stage intervient. Cela consiste dans un premier temps à faire l'état de l'art des différents modèles qui existent sur des images naturelles. Dans un second temps le but est d'adapter le meilleur modèle pour qu'il s'adapte çà des peintures. Et enfin à partir des résultats de ce modèle trouver des applications visuelles et ludiques pour montrer l'intérêt d'un tel modèle.

\par 


\textit{Explication saillance, fixations, saccades, (top-down and bottum-up ?)}\\
Bottom up : regard basé sur des signaux nerveux simple (contrast luminosité...)
Top down : regard basé sur des éléments propre à l'individu et extérieur à l'image (connaissance, tache imposée...)
~\\
Explication sujet :\\
Base de données construite par stagiaires istic (a mettre dans contexte ?)\\
saillance via machine learning\\
Créer application pour mettre en valeur modèle saillance\\
~\\
Pk avoir choisi le sujet\\