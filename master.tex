\documentclass[12pt, openany]{report}

\usepackage[utf8]{inputenc}
\usepackage[T1]{fontenc}
\usepackage[a4paper,left=2cm,right=2cm,top=2cm,bottom=2cm,headheight=16pt]{geometry}
\usepackage{libertine}
\usepackage[pdftex]{graphicx}
\usepackage{totpages}
\usepackage[hidelinks]{hyperref}

%package pour faire des graphs
\usepackage{pgfplots}
\pgfplotsset{width=7cm,compat=1.8}
\pgfmathdeclarefunction{gauss}{2}{%
  \pgfmathparse{1/(#2*sqrt(2*pi))*exp(-((x-#1)^2)/(2*#2^2))}%
}
% \pgfmathdeclarefunction{ease}{}{%
%   \pgfmathparse{-(cos(pi*x)-1)/2}%
% }

\usepackage{helvet}
\renewcommand{\familydefault}{\sfdefault}
\usepackage[english, french]{babel}
\usepackage{subcaption}

\usepackage{setspace}
\onehalfspacing

% Change l'espacement entre deux paragraphes
\setlength{\parskip}{18pt}

% Liste à puces
\frenchbsetup{StandardLists=true}

% Édite style sous-titre figure et tableau
\usepackage[font=it]{caption}

% Change l'en-tête et le pied de page pour tout le document
\usepackage{fancyhdr}
\pagestyle{fancy}

\renewcommand{\chaptermark}[1]{\markboth{\thechapter.\ #1}{}}
\renewcommand{\sectionmark}[1]{\markright{\thesection.\ #1}}

\renewcommand{\headrulewidth}{0.5pt} 
\fancyhead[L]{\bfseries\leftmark}
\fancyhead[C]{}
\fancyhead[R]{\rightmark}

\renewcommand{\footrulewidth}{0.5pt}
\fancyfoot[L]{\textbf{Tugdual Le Pen}}
\fancyfoot[C]{}
\fancyfoot[R]{\thepage\ / \ref{TotPages}}

\fancypagestyle{plain}{ %
    \fancyhf{} % remove everything

    \renewcommand{\headrulewidth}{0pt} 
    \renewcommand{\footrulewidth}{0.5pt}
    \fancyfoot[L]{\textbf{Tugdual Le Pen}}
    \fancyfoot[C]{}
    \fancyfoot[R]{\thepage\ / \ref{TotPages}}
}

% Change le style des parties et sous partiess
\usepackage[explicit]{titlesec}
% change l'espacement au niveau du titre des chapitres
\titlespacing*{\chapter}
  {0pt}%  indent
  {0pt}% space before
  {12pt}% space after

\titlespacing*{\section}
  {0.6cm}%  indent
  {0pt}% space before
  {0pt}% space after

% Définie les couleurs utilisées dans le rapport
\usepackage{color}

\definecolor{darkblue}{rgb}{0.11, 0.30, 0.66}
\definecolor{lightblue}{rgb}{0.18, 0.42, 0.86}
\definecolor{gray}{rgb}{0.4,0.4,0.4}
\definecolor{black}{rgb}{0.0,0.0,0.0}
\definecolor{green}{rgb}{0.4, 0.8, 0.2}

%% Style Chapitre
% -------------------------------------------------------
% Avec numéro
\titleformat{\chapter}[hang] 
    {\fontsize{24pt}{0pt}\selectfont \bfseries}
    {\textcolor{darkblue} 
    {\thechapter. #1}}
    {0pt}
    {\huge}

% Sans numéro
\titleformat{name=\chapter,numberless}[hang] 
    {\fontsize{24pt}{0pt}\selectfont \bfseries}
    {\textcolor{darkblue} 
    {#1}}
    {0pt}
    {\huge}
% -------------------------------------------------------

%% Style Section
% -------------------------------------------------------
% Avec numéroté
\titleformat{\section}[hang]
    {\fontsize{16pt}{0pt}\selectfont \bfseries}
    {\textcolor{lightblue} 
    {\thesection.\ #1}}
    {10pt}
    {\Large}

% Sans numéro
\titleformat{name=\section,numberless}[display] 
    {\fontsize{16pt}{0pt}\selectfont \bfseries}
    {\textcolor{lightblue} 
    {\thechapter.\thesection #1}}
    {10pt}
    {\Large}
% -------------------------------------------------------

% Numérotation des chapitre en chiffre Romain
\renewcommand{\thechapter}{\Roman{chapter}}

% Créer commande pour lien url
\newcommand{\link}[1]{{\color{lightblue}\href{#1}{#1}}}
\newcommand{\linkrename}[2]{{\color{lightblue}\href{#2}{#1}}}

% Éviter les orphelin en début ou fin de page
\widowpenalty=10000 
\clubpenalty=10000 
\raggedbottom

% Nouvelle commande pour ajouter des commentaire sur le rapport
\newcommand{\comment}[1]{\emph{\color{green} \% #1}}



\begin{document}
\selectlanguage{french}
% Édite style sous-titre image et tableau
\makeatletter
\newcommand{\figcapfont}{\itshape} 
\newcommand{\tabcapfont}{\itshape}
\renewcommand{\fnum@figure}{\figcapfont Image \thefigure}
\renewcommand{\fnum@table}{\tabcapfont Tableau \thetable}
\makeatother

\begin{titlepage}
  \begin{center}

    % Logo Esir and Irisa
    % Author and supervisor
    \begin{minipage}{0.45\textwidth}
      \begin{flushleft} \large
        \includegraphics[width=0.9\columnwidth]{datas/logo_esir.jpg}~\\
        \emph{Élève-ingénieur :}\\
        Tugdual Le Pen\\
        Imagerie Numérique\\
        2\up{ème} année du cursus ingénieur\\
        ~\\
        \emph{Tuteur universitaire :}\\
        Pierre Maurel\\
        Enseignant Chercheur
      \end{flushleft}
    \end{minipage}
    \begin{minipage}{0.45\textwidth}
      \begin{flushright} \large
        \vspace{19pt}
        \includegraphics[width=0.9\columnwidth]{datas/logo_irisa.jpg}~\\~\\
        IRISA\\
        263 Avenue du Général Leclerc\\
        35000 RENNES - France\\
        contact@irisa.fr\\
        ~\\        
        \emph{Tuteur d'entreprise :}\\
        Olivier Le Meur\\
        Enseignant Chercheur\\
      \end{flushright}
    \end{minipage}

    \vspace{4cm}

    \textsc{\Huge \textbf{Étude et modèle prédictif de la saillance sur des œuvres d’art}}\\    

    \vfill

    % Bottom of the page
    \begin{minipage}{0.45\textwidth}
      \begin{flushleft}
        \vspace{0.5cm}
        {\large Année universitaire 2019 - 2020}
      \end{flushleft}
    \end{minipage}
    \begin{minipage}{0.45\textwidth}
      \begin{flushright}
        \includegraphics[width=0.9\columnwidth]{datas/logo_univ.png}~\\
      \end{flushright}
    \end{minipage}

  \end{center}
\end{titlepage}

% Actualise le compteur de page
\clearpage
\setcounter{page}{2}

\chapter{Remerciements}
\section{C'est un plaisir}

Merci !
\newpage
Vraiment merci !

\addcontentsline{toc}{chapter}{Résumé}
\chapter*{Résumé}
\par
    Pour valider ma 4\up{ème} année de mon cycle ingénieur en Technologie de 
    l'Information avec spécialité Imagerie Numérique, j'ai effectué un stage 
    d'une durée de six mois dans l'Institut de Recherche en Informatique et 
    Systèmes Aléatoires (IRISA). C'est un laboratoire de recherche impliqué dans 
    le domaine de l'informatique et des technologies de l'information. Il couvre 
    l'ensemble des thématiques de ces domaines, de l’architecture des 
    ordinateurs et des réseaux à l’intelligence artificielle en passant par le 
    génie logiciel, les systèmes distribués et la réalité virtuelle.

\par
    J'ai rejoint plus précisément l'équipe Percept (2018) qui est spécialisée 
    dans le comportement visuel de différentes populations. L'un des projets de 
    cette équipe est d'étudier la saillance dans les peintures. Notamment la capacité 
    de déterminer cette saillance automatiquement au moyen de machine learning.

\par
    Mon objectif est de participer à ce projet et mettre en place des 
    applications qui permettraient de montrer les possibilités d'utilisations de 
    ce genre de programme.

\vspace{40pt}

\selectlanguage{english}
\color{gray}
\par
    To validate my 4\up{th} year of my engineer cycle specializing in Digital 
    Imaging, I did a six-month internship in the Research Institute in Computer 
    Science and Random Systems (IRISA). It is a research laboratory involved in 
    the field of computer science and information technology. It covers all the 
    themes of these fields, from the architecture of computers and networks to 
    artificial intelligence, including software engineering, distributed systems 
    and virtual reality. 

\par
    I joined the team Percept (2018) which specializes in the visual behavior of 
    different populations. One of the projects of this team is to study the 
    saliency in paintings. In particular the ability to determine this salience 
    automatically by means of machine learning. 

\par
    My objective is to participate in this project and set up applications which 
    allow us to show the possibilities of uses of this kind of program.

\selectlanguage{french}
\color{black}

% Renomme "Tables des matières" en "Sommaire"
\renewcommand{\contentsname}{Sommaire}
{\setlength{\parskip}{6pt}
\tableofcontents
}

\chapter{Introduction}

% rapide résumé contexte, explication du sujet et pk avoir choisi le stage

%Intro contexte
\par
La peinture et le regard de l'Homme ont toujours eu un lien étroit. En effet, chaque spectateur regardera un tableau d'une manière différente de son voisin parce que chaque individu a sa propre culture, son propre point de vue... Pourtant, la structure d'une peinture invitera le spectateur à suivre un sens de lecture. Celui-ci sera commun à une grande majorité des spectateurs. Par exemple, un individu qui découvre le tableau de La Joconde pour la première fois regardera presque systématiquement en premier lieu le visage de Mona Lisa et particulièrement ces yeux qui ont un effet particulier. Rares sont les personnes qui commenceront par identifier les éléments du décor en arrière-plan de la peinture.

\begin{figure}[h]
    \centering
    \includegraphics[width=0.3\textwidth]
                    {datas/Mona_Lisa_by_Leonardo_da_Vinci.jpg}
    \caption{\emph{La Joconde}, Léonard de Vinci, 1513}
\end{figure}

%Intro saillance, Irisa et percept
\par
Ce sont l'ensemble de ces éléments qui attirent l'\oe{}il humain qui constituent la saillance visuelle. C'est un élément important pour de nombreux domaines. On pense notamment au domaine du marketing et de la publicité qui doivent créer des affiches ou des spots publicitaires avec pour objectif d'attirer le plus possible l'attention et le regard des consommateurs.

%Intro 
\par
L'étude de la saillance dans la peinture permet d'analyser et de comprendre le regard humain ainsi que toutes les particularités qui en découlent. L'équipe Percept, équipe de recherche du laboratoire de l'IRISA, se penche sur le sujet et cherche à automatiser la détection d'éléments saillants dans les peintures à l'aide de modèles de réseaux de neurones basés sur l'apprentissage machine.

%Explication du sujet de stage
\par
C'est là que mon sujet de stage intervient. Cela consiste dans un premier temps à faire l'état de l'art des différents modèles de saillance qui existent sur des images naturelles. Dans un second temps, le but est de ré-entrainer le meilleur modèle pour qu'il s'adapte à des peintures. Et enfin, à partir des résultats de ce modèle, trouver des applications visuelles et ludiques pour montrer l'intérêt d'un tel modèle.

%Explication comment j'ai trouvé le stage et quelles étaient mes motivations
\par 
Ce stage qui m'a été proposé par Olivier Le Meur correspondait à ce que je recherchais. C'est-à-dire un stage basé sur l'apprentissage machine, qui fait suite à mon projet industriel à l'ESIR qui consistait à générer des visages au moyen de réseau de neurones antagoniste génératif (GAN). Mais aussi un stage varié qui puisse me permettre de me former sur plusieurs compétences différentes.


\chapter{IRISA}

\par
L'IRISA - Institut de Recherche en Informatique et Systèmes Aléatoires - est
aujourd'hui le plus grand laboratoire de recherche français (+ de 850 personnes)
dans le domaine de l'informatique et des technologies de l'information. Il
couvre l'ensemble des thématiques de ces domaines, de l'architecture des
ordinateurs et des réseaux à l'intelligence artificielle en passant par le génie
logiciel, les systèmes distribués et la réalité virtuelle.

\par
L'IRISA, créé en 1975, est issu d'une volonté de collaboration entre huit établissements tutelles pluridisciplinaires : CentraleSupélec, CNRS, ENS Rennes, IMT Atlantique, Inria, INSA Rennes, Université Bretagne Sud, Université de Rennes 1. Il est aujourd'hui dirigé par Jean Marc Jézéquel.

\begin{figure}[h]
    \centering
    \includegraphics[width=0.7\linewidth]{datas/logo_irisa.jpg}
    \caption{Logo de l'IRISA}
\end{figure}

\par
l'IRISA est présent sur 3 sites géographiques au sein du territoire breton (Rennes, Lannion et Vannes). Mon stage s'est déroulé dans les locaux de Rennes.
Le laboratoire est structuré en sept départements scientifiques :
\begin{itemize}
    \item D1 - Systèmes Large Échelle
    \item D2 - Réseaux, Télécommunication et Services
    \item D3 - Architecture
    \item D4 - Langage et génie logiciel
    \item D5 - Signaux et Images numériques, Robotique
    \item D6 - Média et interactions
    \item D7 - Gestion des données et dela connaissance
\end{itemize}


\par
L'équipe PERCEPT du département Média et intéractions est spécialisé dans le comportement visuel de différentes populations.

\chapter{Contexte}
% parler de la saillance, fixation, saccades, dataset
\par
Le stage a commencé par de la documentation par rapport au sujet de stage. N'ayant pas accès à un poste la première semaine de stage, Olivier m'a donné quelques documents en rapport avec le regard et la saillance. Ce sont des domaines plutôt liés à l'anatomie et la psychologie mais qu'il est important de comprendre si l'on veut pouvoir interpréter les résultats en sortie des programmes.
\par
Je vais vous expliquer ici les notions qui permettent de comprendre le regard humain et comment ont peut le mesurer pour l'analyser.

\section{Fixation et saccade}
\par
Le regard est une alternance entre des périodes où l'\oe{}il reste relativement stationnaire, que l'on appelle "\textbf{fixations}", et de courtes périodes de plus grande mobilité, que l'on appelle "\textbf{saccades}"\cite{gaze}. Ce sont des notions qui ont été décrites pour la première fois en 1879 par Javal et Lamare. Il a été possible d'établir des mesures sur le mouvement des yeux deux décades plus tard (Erdmann et Dodge 1898). Ces mesures ont ouvert le champs aux possibilités d'expérimentation sur la psychologie liée au mouvement des yeux. Cela a permis de mieux comprendre le processus d'analyse quand quelqu'un lit, résoud un problème, regarde un film ou quand il regarde une peinture. Par exemple récemment l'équipe Percept à sorti une étude sur le regard des personnes atteintes d'autisme.

\par
Chaque fixation est reliée à un autre point de fixation par une saccade. Quand on mesure le regard d'une personne dans le temps, on obtient donc une succession de fixations et de saccades qui forme une chaine. On appelle cela le \textbf{chemin visuel}. C'est en analysant le chemin visuel que l'on est capable de comprendre comment un individu regarde une peinture ou tout autre élément visuel.

\begin{figure}[!ht]
    \centering
    \includegraphics[width=0.7\linewidth]{datas/exemple_scanpaths2.png}
    \caption{Exemple de chemins visuels de 3 observateurs différents.}
    \label{ex_scanpath}
\end{figure}

\par
Sur l'image \ref{ex_scanpath} (\emph{Avenue of trees in a small town}, A. Sisley, 1866) on peut voir l'exemple d'une représentation des chemins visuels de trois observateurs différents distingués chacun par une couleur différentes. Sur cette représentation chaque fixation est représentée par un cercle numéroté qui correspond à son positionnement dans le parcours du regard. La taille des cercles dépend de la durée de la fixation en question. Ici chaque fixation est reliée par un trait qui représente la saccade entre 2 fixations.

\section{Saillance et carte de saillance}
\par
On remarque bien sur l'exemple précédent que chaque individu aborde la peinture avec un regard différent d'un autre. Cependant il est très important de noter qu'il y a des similarités dans les zones regardées. Il y a des endroits de la peintures que les trois observateurs ont regardé, comme le bout du chemin ou l'arbre de gauche au premier plan, tandis que d'autres endroits sont ignorés, les bordures de la peinture entre autre. Ce sont ces zones plus souvent regardées que l'on appellera des zones saillantes.

\par
La \textbf{saillance} est donc la notion qui définie qu'un élément est facilement remarqué. Elle existe aussi dans le domaine sonore ou linguistique. Ici c'est évidemment la saillance visuelle qui nous intéresse. Un élément dit saillant est donc un élément qui dénote du reste de l'oeuvre et qui attirera le regard de l'observateur de manière quasi-systématique. Les critères qui définisse la saillance d'un objet sont régies par deux fateurs important. Le facteur "\textbf{Bottom-up}" basé sur les information simple comme les couleurs, le contraste ou la luminosité. Le facteur "\textbf{Top-down}" lui est basé sur des informations propres à l'observateur. Il peut être influencé par une tâche à accomplir, par sa culture, ses connaissances, son âge, ... C'est à partir de tous ces éléments que l'on peut déduire une forte connexion entre le mouvement des yeux, ou le chemin visuel, et la saillance d'une peinture.

\par
De cette relation on va pouvoir, à partir du chemin visuel d'un observateur, générer une \textbf{carte de saillance} (voir image \ref{ex_saliency_map}). Celle-ci va nous permettre de mettre en évidence les éléments saillants d'une image. \textit{Regarder code pour voir comment c'est fait globalement}. 

\begin{figure}[!ht]
    \centering
    \includegraphics[width=0.7\linewidth]{datas/exemple_saliency_map.png}
    \caption{Exemple de carte de saillance}
    \label{ex_saliency_map}
\end{figure}

\par
Si on superpose la carte de saillance et la peinture associée on peut facilement voir quels sont les éléments saillants de l'oeuvre (voir image \ref{saliency_map_transparency}). Ici la carte de saillance nous révèle que le village au bout du chemin, les deux personnages en bas à gauche et la végétation au premier plan sont les éléments saillants de l'oeuvre.

\begin{figure}[!ht]
    \centering
    \includegraphics[width=0.7\linewidth]{datas/exemple_saliency_map_transparency.png}
    \caption{Carte de saillance et peinture superposées}
    \label{saliency_map_transparency}
\end{figure}

\par
Mon but lors de ce stage sera d'entrainer un réseau de neurones capable de générer une carte de saillance avec comme seul entrée une peinture. 

\bibliographystyle{plain}
\addcontentsline{toc}{chapter}{Bibliographie}
\begin{thebibliography}{99}

	\bibitem{irisa}
	  Site de l'IRISA - Présentation du laboratoire\\
    \link{https://www.irisa.fr/fr/page/recherche-innovation-sciences-technologies-du-numerique}
      
  \bibitem{percept}
	  Site de l'IRISA - Présentation de l'équipe PERCEPT\\
    \link{https://www.irisa.fr/fr/equipes/percept}

  \bibitem{gaze}
    \emph{Art, Aesthetics, and the brain}, 2015\\
    par J.P. Huston, M. Nadal, F. Mora, L.F. Agnati et C.J. Cela-Conde

  \bibitem{benchmark_MIT}
    \emph{MIT saliency benchmark}, 2015\\
    par Bylinskii Z, Judd T, Borji A, Itti L, Durand F, Oliva A, et al.\\
    \link{http://saliency.mit.edu/}

  \bibitem{eaktalab}
    Projet "Eye-tracking and Comics" de Eakta Jain\\
    \link{https://jainlab.cise.ufl.edu/comics.html}

  \bibitem{kenburns}
    \emph{Predicting Moves-on-Stills for Comic Artusing Viewer Gaze Data}, 2015\\
    E. Jain, Y. Sheikh, J. Hodgins\\
    \link{https://jainlab.cise.ufl.edu/documents/motion-comics-cga2015.pdf}

  \bibitem{kenburns3D}
    \emph{3D Ken Burns Effect from a Single Image}, 2019\\
    S. Niklaus, L. Mai, J. Yang, F. Liu\\
    \link{https://arxiv.org/pdf/1909.05483.pdf}

  \bibitem{saccadicmodel}
    \emph{Saccadic model of eye movements for free-viewing condition}, 2015\\
    O. Le Meur, Z. Liu\\
    \link{https://www.sciencedirect.com/science/article/pii/S0042698915000504}

  \bibitem{segmentationcomics}
    \emph{Creating Segments and Effects on Comics by Clustering Gaze Data}, 2017\\
    I. Thirunarayanan, K. Khetarpal, S. Koppal, O. Le Meur, J. Shea, E. Jain\\
    \link{https://jainlab.cise.ufl.edu/documents/REQGazeComics\_preprint.pdf}

  \bibitem{neuraltalk2}
    Neuraltalk2 code\\
    A. Karpathy\\
    \link{https://github.com/karpathy/neuraltalk2}

\end{thebibliography}


\appendix
\addcontentsline{toc}{chapter}{Annexes}
\chapter*{Annexes}


\end{document}

% Plan :
% Page de garde -- Done
% Remerciements -- Done
% Résumé (anglais français) -- Done
% Sommaire -- Done
% Introduction
%%% rapide résumé contexte, entreprise, explication du sujet et pk avoir choisi le stage
% Présentaion Irisa et mission de stage
% Analyse du problème et de son contexte / étude de l'existant / méthode et plan de travail
%%% Introduction à la saillance et au machine learning
% Solution proposée / Travail réalisé / Résultats développés
%%% prise en main de python et de la saillance : video fondu et scanpath
%%% diapo pour journée du patrimoine irisa
%%% puis benchmark des differents algo
%%% puis finetuning de SAM-RESNET (the best)
%%% appli : ken burns video + ...
% évaluation / Interprétation des résultats / Retour d'expérience / Bilan
%%% finetuning a permis de faire un papier (accepter par plosone ?)
% Conclusion
% Biblio
% Annexes


