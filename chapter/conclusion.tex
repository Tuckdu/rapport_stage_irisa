\chapter{Conclusion}
\par
Au cours de ces 6 mois de stage au sein de l'IRISA et plus particulièrement l'équipe Percept, j'ai pu découvrir le monde de la recherche. Une expérience unique qui m'a permis de me faire une idée de ce qui m'attend si je me lance dans le développement d'une thèse.
\par
J'ai aussi pu découvrir le domaine de la saillance. Je pense que pour un étudiant qui a pour but de travailler dans l'imagerie numérique il est important de savoir coment fonctionne le regard humain. On a bien sûr déjà eu des cours sur ce sujet à l'ESIR mais ici j'ai pu approfondir mes connaissances sur ce sujet.
\par
J'ai beaucoup aimé l'échange qu'il y a entre les différentes personnes au sein de l'équipe. Dès mon arrivée je me suis senti intégré à ce groupe de travail. Je l'ai aussi ressenti lors des réunions où tous les membres de l'équipe devaient présenter leur projet moi y compris. C'est un très bon moyen de comprendre ce que font les collègues et à l'inverse expliquer mon sujet. Cela permet donc par la suite que chacun puisse proposer des idées ou des solutions aux problèmes d'autres personnes qui bloquent dans leur projet.
\par
J'ai vraiment apprécié la liberté et l'autonomie qui m'a été donné durant ce stage. Les réunions et discussions qu'on a pu avoir avec Olivier pour déterminer quelles seraient les futur étapes du projet m'ont permis de me mettre à l'aise et de sentir que mon avis comptait.
\par
Au niveau technique, j'ai beaucoup appris en programmation et particulièrement en Python et en \LaTeX. Pour Python je n'avais jamais réalisé de projet aussi concret et qui soit ensuite mis à la disposition de l'équipe. \LaTeX\ est un langage que j'ai eu la chance d'avoir le temps de maitriser suffisament pour obtenir des documents sobres, professionnels et automatiques.
\par
Ce fut aussi intéressant de travailler avec l'apprentissage profond qui est aujourd'hui présent partout. Même si j'avais déjà travaillé sur un projet de réseaux de neurones à l'ESIR, j'ai pu renforcé mes connaissances sur le sujet.