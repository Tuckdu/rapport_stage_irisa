\chapter{IRISA}

\par
L'IRISA - Institut de Recherche en Informatique et Systèmes Aléatoires - est
aujourd'hui le plus grand laboratoire de recherche français (+ de 850 personnes)
dans le domaine de l'informatique et des technologies de l'information. Il
couvre l'ensemble des thématiques de ces domaines, de l'architecture des
ordinateurs et des réseaux à l'intelligence artificielle en passant par le génie
logiciel, les systèmes distribués et la réalité virtuelle.

\par
L'IRISA, créé en 1975, est issu d'une volonté de collaboration entre huit établissements tutelles pluridisciplinaires : CentraleSupélec, CNRS, ENS Rennes, IMT Atlantique, Inria, INSA Rennes, Université Bretagne Sud, Université de Rennes 1. Il est aujourd'hui dirigé par Jean Marc Jézéquel.

\begin{figure}[h]
    \centering
    \includegraphics[width=0.7\linewidth]{datas/logo_irisa.jpg}
    \caption{Logo de l'IRISA}
\end{figure}

\par
l'IRISA est présent sur 3 sites géographiques au sein du territoire breton (Rennes, Lannion et Vannes). Mon stage s'est déroulé dans les locaux de Rennes.
Le laboratoire est structuré en sept départements scientifiques :
\begin{itemize}
    \item D1 - Systèmes Large Échelle
    \item D2 - Réseaux, Télécommunication et Services
    \item D3 - Architecture
    \item D4 - Langage et génie logiciel
    \item D5 - Signaux et Images numériques, Robotique
    \item D6 - Média et interactions
    \item D7 - Gestion des données et dela connaissance
\end{itemize}


\par
L'équipe PERCEPT du département Média et intéractions est spécialisé dans le comportement visuel de différentes populations.