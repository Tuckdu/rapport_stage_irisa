\chapter{Applications}

\par
Les applications possibles à partir des cartes de saillance sont illimitées. Eakta Jain, professeur assistante à l'université de de Floride, a déjà travailler avec Olivier et a pu nous apporter son expérience car elle a déjà réalisé des projets d'applications basés sur le mouvement du regard \cite{eaktalab}. L'intérêt qu'apportent ses projets c'est le domaine artistique des comics qui rejoint celui de la peinture.

\section{Recherche d'application}

\par
Afin de déterminer quelle application il serait intéressant de développer, il faut faire une recherche de ce qui existe déjà ou inventer autre chose. Je vais présenter ici les différentes pistes qui ont été envisagées et expliquer notre choix finale.

\par
Pour ajouter du mouvement, il existe des logiciels qui permettent de générer des plotagraphes. Basé sur les cinématographes, qui sont des mélanges d'images et de vidéos, les plotagraphes permettent d'ajouter du mouvement à partir de l'image seule. En revanche il faut ajuster à la main les zones à mouvoir donc difficilement automatisable et le résultat n'est impressionnant que sur les fluides (eau, nuages, fumée...).

\par
Eakta a un projet de segmentation des différentes parties d'un comic pour animer les personnes qui parle et ajouter du son \cite{segmentationcomics}. La segmentation est obtenue à partir du chemin visuel aquis par occulométrie. Une idée intéressante mais difficlement adaptable à toutes les peintures car trop détaillée.

\par
Un autre projet d'Eakta appelé "Predicting Moves-on-Stills for Comic Art using Viewer Gaze Data" \cite{kenburns} ajoute un effet de Ken Burns sur des pages de comics en fonction des prédictions du mouvement du regard du spectateur. Cet un effet intéressant et facile à mettre en place. C'est la piste vers laquelle on se dirigera et que l'on détaillera dans la partie suivante. Il existe aussi une version 3D de cet effet \cite{kenburns3D} qui est très impressionant mais est compliqué a mettre en place.

\newpage
\section{Effet Ken burns}

\par
L'effet Ken Burns est représenté par un mouvement de caméra (panoramique, zoom ou rotation) sur une image fixe. L'ajout de mouvement et d'animation permet de garder l'attention du spectateur. C'est un effet qui est très régulièrement utilisé dans les documentaires, dans les journeaux ou tout simplement dans les diaporamas photos. 

\par
J'ai donc décidé de partir sur un effet Ken Burns comme projet d'application. Mon but ici est que le mouvement de la caméra suive celui d'un \oe{}il humain. On est capable de généré un chemin visuel à partir d'une carte de saillance grâce à un modèle saccadique. J'ai pu en utilisé un créé par Olivier Le Meur \cite{saccadicmodel} disponible sur le gitlab de l'équipe Percept. Ainsi à partir d'une peinture on obtient une carte de saillance grâce au modèle de saillance, puis le chemin visuel grâce au modèle saccadique et enfin un effet Ken Burns qui suit le regard humain.

\par
