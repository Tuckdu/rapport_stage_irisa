\chapter{Introduction}

% rapide résumé contexte, explication du sujet et pk avoir choisi le stage

%Intro contexte
\par
La peinture et le regard de l'Homme ont toujours eu un lien étroit. En effet, chaque spectateur regardera un tableau d'une manière différente de son voisin parce que chaque individu a sa propre culture, son propre point de vue... Pourtant, la structure d'une peinture invitera le spectateur à suivre un sens de lecture. Celui-ci sera commun à une grande majorité des spectateurs. Par exemple, un individu qui découvre le tableau de La Joconde pour la première fois regardera presque systématiquement en premier lieu le visage de Mona Lisa et particulièrement ces yeux qui ont un effet particulier. Rares sont les personnes qui commenceront par identifier les éléments du décor en arrière-plan de la peinture.

\begin{figure}[h]
    \centering
    \includegraphics[width=0.3\textwidth]
                    {datas/Mona_Lisa_by_Leonardo_da_Vinci.jpg}
    \caption{\emph{La Joconde}, Léonard de Vinci, 1513}
\end{figure}

%Intro saillance, Irisa et percept
\par
Ce sont l'ensemble de ces éléments qui attirent l'\oe{}il humain qui constituent la saillance visuelle. C'est un élément important pour de nombreux domaines. On pense notamment au domaine du marketing et de la publicité qui doivent créer des affiches ou des spots publicitaires avec pour objectif d'attirer le plus possible l'attention et le regard des consommateurs.

%Intro 
\par
L'étude de la saillance dans la peinture permet d'analyser et de comprendre le regard humain ainsi que toutes les particularités qui en découlent. L'équipe Percept, équipe de recherche du laboratoire de l'IRISA, se penche sur le sujet et cherche à automatiser la détection d'éléments saillants dans les peintures à l'aide de modèles de réseaux de neurones basés sur l'apprentissage machine.

%Explication du sujet de stage
\par
C'est là que mon sujet de stage intervient. Cela consiste dans un premier temps à faire l'état de l'art des différents modèles de saillance qui existent sur des images naturelles. Dans un second temps, le but est de ré-entrainer le meilleur modèle pour qu'il s'adapte à des peintures. Et enfin, à partir des résultats de ce modèle, trouver des applications visuelles et ludiques pour montrer l'intérêt d'un tel modèle.

%Explication comment j'ai trouvé le stage et quelles étaient mes motivations
\par 
Ce stage qui m'a été proposé par Olivier Le Meur correspondait à ce que je recherchais. C'est-à-dire un stage basé sur l'apprentissage machine, qui fait suite à mon projet industriel à l'ESIR qui consistait à générer des visages au moyen de réseau de neurones antagoniste génératif (GAN). Mais aussi un stage varié qui puisse me permettre de me former sur plusieurs compétences différentes.
